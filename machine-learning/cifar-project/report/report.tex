\documentclass[12pt,abstract=true]{scrartcl}

\usepackage{tikz}
\usepackage{array}
\usepackage{pifont}
\usepackage{engord}
\usepackage{listings}
\usepackage{titlesec}
\usepackage{fancyhdr}
\usepackage{fancyvrb}
\usepackage{multicol}
\usepackage{pgfplots}
\usepackage{algorithm}
\usepackage{algpseudocode}
\usepackage[plain]{fancyref}
\usepackage[unicode]{hyperref}
\usepackage{graphicx,tabularx}
\usepackage[shortlabels]{enumitem}
%\usepackage[retainorgcmds]{IEEEtrantools}
\usepackage{amsmath,amssymb,amsthm,amscd,amstext}
\usepackage{mathtools}
\usepackage{exscale}
\usepackage{relsize}
\usepackage{mathrsfs}
\usepackage{tocloft}

\usepackage{setspace}
\usepackage{lastpage}
\usepackage{extramarks}
\usepackage{chngpage}

\usepackage{charter}

%\pgfplotsset{compat=1.6}
\usetikzlibrary{arrows}
\usetikzlibrary{patterns}
\usetikzlibrary{positioning}
\usetikzlibrary{automata}
\usetikzlibrary{plotmarks}
\usepackage{tikz-3dplot}

\tikzset{ alert/.style={ very thick, draw=red!80, fill=red!20 } }
\tikzset{ fancy/.style={ very thick, draw=blue!80, fill=blue!20 } }
\tikzset{
  dot/.style={
    very thick,circle,draw=black,fill=black,inner sep=0,minimum size=5pt
  }
}
\tikzset { every state/.style={fancy} }
\tikzset {
  general/.style = {
    shorten >=2pt, node distance=2.5cm, on grid, thick,>=stealth', auto
  }
}
\newcommand{\mathhuge}[1]{\mathlarger{\mathlarger{\mathlarger{#1}}}}
\newcommand{\mathtiny}[1]{\mathsmaller{\mathlarger{\mathlarger{#1}}}}

\renewcommand\cftsecfont{\normalsize}
\renewcommand\cftsecpagefont{\normalsize}
\renewcommand\cftsecafterpnum{\par}
\setlength{\cftbeforesecskip}{0.2em}

\renewcommand\cftsubsecfont{\small}
\renewcommand\cftsubsecpagefont{\small}
\renewcommand\cftsubsecafterpnum{\par}

\newcommand\upper[1]{\textsuperscript#1}
\numberwithin{equation}{section}
\mathtoolsset{centercolon}
\bibliographystyle{plain}

% Theorem environment <<<
\theoremstyle{definition}   \newtheorem{definition}{Definition}[section]
\theoremstyle{plain}        \newtheorem{theorem}{Theorem}[section]
\theoremstyle{plain}        \newtheorem{observation}{Observation}[section]
\theoremstyle{plain}        \newtheorem{fact}{Fact}[section]
\theoremstyle{plain}        \newtheorem{claim}{Claim}[section]
\theoremstyle{plain}        \newtheorem{lemma}[theorem]{Lemma}
\theoremstyle{plain}        \newtheorem{corollary}[theorem]{Corollary}
\theoremstyle{remark}       \newtheorem{example}{Example}[section]
\theoremstyle{remark}       \newtheorem{remark}{Remark}[section]
\newcommand*{\fancyrefdeflabelprefix}{def}
\newcommand*{\fancyrefthmlabelprefix}{thm}
\newcommand*{\fancyreflemlabelprefix}{lem}
\newcommand*{\fancyrefcorlabelprefix}{cor}
\newcommand*{\fancyrefalglabelprefix}{alg}
\newcommand*{\fancyreflnlabelprefix}{ln}
\frefformat{plain}{\fancyreflnlabelprefix}{line #1}
\Frefformat{plain}{\fancyreflnlabelprefix}{Line #1}
\frefformat{plain}{\fancyrefalglabelprefix}{algorithm (#1)}
\Frefformat{plain}{\fancyrefalglabelprefix}{Algorithm (#1)}
\frefformat{plain}{\fancyrefdeflabelprefix}{definition (#1)}
\Frefformat{plain}{\fancyrefdeflabelprefix}{Definition (#1)}
\frefformat{plain}{\fancyrefthmlabelprefix}{theorem (#1)}
\Frefformat{plain}{\fancyrefthmlabelprefix}{Theorem (#1)}
\frefformat{plain}{\fancyreflemlabelprefix}{lemma (#1)}
\Frefformat{plain}{\fancyreflemlabelprefix}{Lemma (#1)}
\frefformat{plain}{\fancyrefcorlabelprefix}{corollary (#1)}
\Frefformat{plain}{\fancyrefcorlabelprefix}{Corollary (#1)}
%>>>
% Alter some LaTeX Float for better treatment of figures: <<<
% See p.105 of "TeX Unbound" for suggested values.
% See pp. 199-200 of Lamport's "LaTeX" book for details.
%   General parameters, for ALL pages:
\renewcommand{\topfraction}{0.9}	% max fraction of floats at top
\renewcommand{\bottomfraction}{0.8}	% max fraction of floats at bottom
\setcounter{topnumber}{2}
\setcounter{bottomnumber}{2}
\setcounter{totalnumber}{4}     % 2 may work better
\setcounter{dbltopnumber}{2}    % for 2-column pages
\renewcommand{\dbltopfraction}{0.9}	% fit big float above 2-col. text
\renewcommand{\textfraction}{0.07}	% allow minimal text w. figs
\renewcommand{\floatpagefraction}{0.7}	% require fuller float pages
\renewcommand{\dblfloatpagefraction}{0.7}	% require fuller float pages

\newcommand{\scorebest}{$83.59\%$}
\newcommand{\scorebasic}{$78.37\%$}
\newcommand{\rankbest}{$9$}

\allowdisplaybreaks[4]

\begin{document}

\begin{center} %<<< Title
	\textsc{
		\LARGE Machine Learning Project Report
	} \\[1em]
	
	\textrm{
		{\Large CIFAR-10}
	} \\[2em]
	
	\textit{
		Kaggle ID: bug-fixer
	} \\[1em]
	
	\normalsize
		Yuxing Zhang\upper{1}, Qipeng Liu\upper{2},
		Yuan Deng\upper{3},	Qiwei Feng\upper{4}\\[1.5em]
	\small
	\begin{tabular}{*{2}{>{\centering}p{.32\textwidth}}}
		\upper{1}\small\href{mailto:saintmakarovski@gmail.com}{saintmakarovski@gmail.com} &%
		\upper{2}\href{mailto:lqp1831@gmail.com}{lqp1831@gmail.com}
		\tabularnewline
		\tabularnewline
		\upper{3}\href{mailto:deng1227yuan@163.com}{deng1227yuan@163.com} &
		\upper{4}\href{mailto:caiwaifung@gmail.com}{caiwaifung@gmail.com}
		%\upper{4}\href{mailto:lqp1831@gmail.com}{lqp1831@gmail.com}% \tabularnewline
	\end{tabular} \\[1.5em]

	\normalsize Institute for Interdisciplinary Information Sciences,
	Tsinghua University \\[2em]
	
	\normalsize
\end{center} \par %>>>

\tableofcontents
\newpage

\section{Introduction}
	We worked on the CIFAR-10 competetion on
		\href{http://www.kaggle.com/c/cifar-10}{www.kaggle.com}.
	A series of algorithms were used by us and we get the best result of \scorebest,
		which toke the \rankbest-th place on Kaggle.
	(Here is a link of the
		\href{http://www.kaggle.com/c/cifar-10/leaderboard}{ranklist}.)
		
	The main method we use is K-Means clustering.
	We use linear SVM for later model training and query answering.
	We also use a lot of libraries to accelerate our program.
	The details of the project would be discuss in later sections.
	
	%\subsection{About Makarovski}
		%Well I think that you should notice Makarovski.
		%{\color{red}TODO: Mention some junk!}
		

\section{Algorithms}

	\subsection{K-Means Clustering}
		K-means clustering is a method of vector quantization, originally from signal processing, that is popular for cluster analysis in data mining. k-means
        clustering aims to partition n observations into k clusters in which each observation belongs to the cluster with the nearest mean, serving as a prototype of the cluster.                
        
	
	\subsection{Feature Extracting}
		The algorithm we used to extracting features is nearly the same
			as A. Coates done in his 2010 NIPS paper \cite{singlelayer}.
		It contains two parts.
		First, we would calculate the "centroids" of the pictures by extracting
			random patches from them and run K-Means on these patches.
		Second, we would compute the feature vector of each picture using the centroids.
		After these, we can get the features of all picture and we can label them for furthur usage.

		We define a patch as a $d*d$ sub-picture of a large $h*w$ picture
			(then the picture has $(h-d+1)*(w-d+1)$ patches totally).
		For a three-channel (RGB) picture, a patch of it can be described as a vector in $R^{3d^2}$.
		First step of the algorithm is to generate the "centroids" of the data set.
		A centroid is also a vector in $R^{3d^2}$ and the centroids are the
			"most common patches" in all the pictures.
		We randomly extract $n$ patches from the picture set to get matrix $P$,
			and then run the K-Means process to generate $k$ centroids $C_1,...,C_k$.
		The K-Means algorithm contains $t$ iterations.
		In each iteration, we just find the closest centroid for each patch in $P$,
			then assign the patch to the centroid.
		For each centroid $C_i$,
			we take all patches that assigned to this centroid in the current iteration,
			and modify the centroid into a new one $C'_i$ which is the mean of all these patches.
		After running such iteration for $t$ rounds,
			our centroids are supposed to well described the 'common features' of the patches.

		Next, we can extract feature from a picture using the centroids.
		Suppose the picture has dimension of $h*w$.
		Know that it has $(h-d+1)*(w-d+1)$ patches.
		For one patch $p\in R^{3d^2}$, we can map it to a $R^k$ vector $f(p)$,
		where
			$$ f_i(p) = \max \{0, \mu - z_i\}, $$
		and $z_i = ||p-C^{(i)}||_2$, $\mu = (\sum_i z_i) / k$.
		Thus we have $(h-d+1)*(w-d+1)$ vectors of dimension $k$.
		We just divide the $(h-d+1)*(w-d+1)$ grids into four equal parts,
			each contains $(h-d+1)*(w-d+1)/4$ grids.
		By summing up all the vectors in each part,
			we get $4$ vectors in $R^k$ which can be merge into a $4k$-dimension vector.
		This vector is used as the feature of this picture.
	
	\subsection{SVM Classifier}
		In machine learning, support vector machines (SVMs, also support vector networks) are supervised learning models with associated learning algorithms
        that analyze data and recognize patterns, used for classification and regression analysis. Given a set of training examples, each marked as belonging to one of two categories, an SVM training algorithm builds a model that assigns new examples into one category or the other, making it a non-probabilistic binary linear classifier.

\section{Implements}
    
	\subsection{Feature Extracting}
		All codes for feature extracting are written in C++.
		The codes are shown in /codes/.
		We use a lot of libraries including MKL (Intel Math Kernel Library)
			for matrix multiplying and eigen-value calculating,
			and OpenCV for image processing.
		Moreover, we use MPI as the framework of the project
			in order to run the algorithm parallelly.
		
		\subsubsection{MPI Framework}
			The Message Passing Interface Standard (MPI) is a message passing library standard.
			We used Microsoft MPI (MS-MPI),
				which is a Microsoft implementation of the MPI standard,
				to develop and runn parallel applications on the Windows platform.
			For the K-Means clustering part,
				we communicate between nodes while summing the centroid matrix,
				thus the centroids can be calculating quickly.
			For the feature extracting part,
				we just assign the extracting tasks to each node
				and then the nodes extract them respectively.
			The parallel implementation truly accelerate the program.
			It took us about half an hour to get the features,
				while it takes days if only one machine is used.
		
		\subsubsection{MKL Library}
			The Intel Math Kernel Library (MKL) is a
				fast and widely used math library for Intel and compatible processors.
			We use this library for matrix multiplication and computing eigen-vectors.
			It is so fast that it takes only about $1/50$ times of hand-written version.
			Since we have met a lot of problems	while using this library,
				it is still a good tool for basic math computations.
				
	\subsection{SVM Training and Testing}
		All our SVM codes are written in MATLAB.
		The code are written by standford as demo of their papers \cite{rf}\cite{singlelayer}.
		It just write a loss function and use the minFunc library
			to minimize the function,
			thus get the support vectors as the model.
		Since CIFAR-10 has 10 categories,
			it just train $\binom{10}{2}$ 1-vs-1 models
			in order to get the most-possible label for a picture.
		MATLAB use MKL as its kernel thus it is fast and reliable.
		We like it.

\section{Results}
	\subsection{Most Basic Results}
		We run the code and train SVM with $C\in[1,100000]$,
			and get results from $70\%$ to about $78\%$.
		In this short range,
			the performance increase when $C$ increase.
		The best result, \scorebasic, comes from $C=100000$.
	
	\subsection{Major Results}
		Since we have run the codes for several times with different parameters,
			we get a number of result lists.
		An idea is that, for a test picture, we can get the major answer from
			these basic answers.
		We do this and adjust some parameters, and get our
			best result, \scorebest.

%\section{Conclusion}
		%{\color{red}TODO: Mention some junk!}
	

\bibliography{report}
\end{document}

