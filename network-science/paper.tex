\documentclass[a4,12pt,abstract=true]{scrartcl}

%<<< Preamble
\usepackage{tikz}
\usepackage{array}
\usepackage{pifont}
\usepackage{engord}
\usepackage{listings}
\usepackage{titlesec}
\usepackage{fancyhdr}
\usepackage{fancyvrb}
\usepackage{multicol}
\usepackage{pgfplots}
\usepackage{algorithm}
\usepackage{algpseudocode}
\usepackage[plain]{fancyref}
\usepackage[unicode]{hyperref}
\usepackage{graphicx,tabularx}
\usepackage[shortlabels]{enumitem}
\usepackage[retainorgcmds]{IEEEtrantools}
\usepackage{amsmath,amssymb,amsthm,amscd,amstext}
\usepackage{mathtools}
\usepackage{exscale}
\usepackage{relsize}
\usepackage{mathrsfs}
\usepackage{tocloft}

\usepackage{setspace}
\usepackage{lastpage}
\usepackage{extramarks}
\usepackage{chngpage}

\usepackage{lmodern}
\usepackage{mathpazo}

\pgfplotsset{compat=1.6}
\usetikzlibrary{arrows}
\usetikzlibrary{patterns}
\usetikzlibrary{positioning}
\usetikzlibrary{automata}
\usepackage{tikz-3dplot}

\tikzset{ alert/.style={ very thick, draw=red!80, fill=red!20 } }
\tikzset{ fancy/.style={ very thick, draw=blue!80, fill=blue!20 } }
\tikzset{
  dot/.style={
    very thick,circle,draw=black,fill=black,inner sep=0,minimum size=5pt
  }
}
\tikzset { every state/.style={fancy} }
\tikzset {
  general/.style = {
    shorten >=2pt, node distance=2.5cm, on grid, thick,>=stealth', auto
  }
}
\newcommand{\mathhuge}[1]{\mathlarger{\mathlarger{\mathlarger{#1}}}}
\newcommand{\mathtiny}[1]{\mathsmaller{\mathlarger{\mathlarger{#1}}}}

\renewcommand\cftsecfont{\normalsize}
\renewcommand\cftsecpagefont{\normalsize}
\renewcommand\cftsecafterpnum{\par}
\setlength{\cftbeforesecskip}{0.3em}

\renewcommand\cftsubsecfont{\small}
\renewcommand\cftsubsecpagefont{\small}
\renewcommand\cftsubsecafterpnum{\par}

\newcommand\upper[1]{\textsuperscript#1}
\numberwithin{equation}{section}
\mathtoolsset{centercolon}
\bibliographystyle{plain}

% Theorem environment <<<
\theoremstyle{definition}   \newtheorem{definition}{Definition}[section]
\theoremstyle{plain}        \newtheorem{theorem}{Theorem}[section]
\theoremstyle{plain}        \newtheorem{observation}{Observation}[section]
\theoremstyle{plain}        \newtheorem{fact}{Fact}[section]
\theoremstyle{plain}        \newtheorem{claim}{Claim}[section]
\theoremstyle{plain}        \newtheorem{lemma}[theorem]{Lemma}
\theoremstyle{plain}        \newtheorem{corollary}[theorem]{Corollary}
\theoremstyle{remark}       \newtheorem{example}{Example}[section]
\theoremstyle{remark}       \newtheorem{remark}{Remark}[section]
\newcommand*{\fancyrefdeflabelprefix}{def}
\newcommand*{\fancyrefthmlabelprefix}{thm}
\newcommand*{\fancyreflemlabelprefix}{lem}
\newcommand*{\fancyrefcorlabelprefix}{cor}
\newcommand*{\fancyrefalglabelprefix}{alg}
\newcommand*{\fancyreflnlabelprefix}{ln}
\frefformat{plain}{\fancyreflnlabelprefix}{line #1}
\Frefformat{plain}{\fancyreflnlabelprefix}{Line #1}
\frefformat{plain}{\fancyrefalglabelprefix}{algorithm (#1)}
\Frefformat{plain}{\fancyrefalglabelprefix}{Algorithm (#1)}
\frefformat{plain}{\fancyrefdeflabelprefix}{definition (#1)}
\Frefformat{plain}{\fancyrefdeflabelprefix}{Definition (#1)}
\frefformat{plain}{\fancyrefthmlabelprefix}{theorem (#1)}
\Frefformat{plain}{\fancyrefthmlabelprefix}{Theorem (#1)}
\frefformat{plain}{\fancyreflemlabelprefix}{lemma (#1)}
\Frefformat{plain}{\fancyreflemlabelprefix}{Lemma (#1)}
\frefformat{plain}{\fancyrefcorlabelprefix}{corollary (#1)}
\Frefformat{plain}{\fancyrefcorlabelprefix}{Corollary (#1)}
%>>>
% Alter some LaTeX Float for better treatment of figures: <<<
% See p.105 of "TeX Unbound" for suggested values.
% See pp. 199-200 of Lamport's "LaTeX" book for details.
%   General parameters, for ALL pages:
\renewcommand{\topfraction}{0.9}	% max fraction of floats at top
\renewcommand{\bottomfraction}{0.8}	% max fraction of floats at bottom
\setcounter{topnumber}{2}
\setcounter{bottomnumber}{2}
\setcounter{totalnumber}{4}     % 2 may work better
\setcounter{dbltopnumber}{2}    % for 2-column pages
\renewcommand{\dbltopfraction}{0.9}	% fit big float above 2-col. text
\renewcommand{\textfraction}{0.07}	% allow minimal text w. figs
\renewcommand{\floatpagefraction}{0.7}	% require fuller float pages
\renewcommand{\dblfloatpagefraction}{0.7}	% require fuller float pages

\allowdisplaybreaks[4]

%>>>
%>>>

\begin{document}

\begin{center} %<<< Title
	\Large \textbf{\textsf{
		Metrics and Algorithms for Evolving Social Networks}} \\[1.2em]
	\normalsize 
		Pufan He\upper{1}, Yichao Zhou\upper{2}, Qiwei Feng\upper{3},
		Yu Yan\upper{4}, Qingyang Li\upper{5} and Xin Xing\upper{6}\\[1em]
	\small
	\begin{tabular}{*{3}{>{\centering}p{.3\textwidth}}}
		\upper{1}\small\href{mailto:hpfdf@126.com}{hpfdf@126.com} &%
		\upper{2}\href{mailto:broken.zhou@gmail.com}{broken.zhou@gmail.com} &%
		\upper{3}\href{mailto:gdfqw93@163.com}{gdfqw93@163.com} \tabularnewline
		\upper{4}\href{mailto:whyvine@hotmail.com}{whyvine@hotmail.com} &
		\upper{5}\href{mailto:591527324@qq.com}{591527324@qq.com} &
		\upper{6}\href{mailto:823291634@qq.com}{823291634@qq.com}
	\end{tabular} \\[1em]

	\small Institute for Interdisciplinary Information Sciences,
	Tsinghua University \\[1.5em]
\end{center} \par %>>>

\begin{abstract}
We address the dynamic properties of typical social networks, in which person
is considered as vertex while relation is considered as edge. We suppose the
evolution of a social network overtime only includes bidirectional new
connections between existing or newly created nodes, and we assume the graph
will evolve as a typical social network. We study the centrality and activity
metrics for nodes in such graphs by regulate a series of reasonable properties
the metrics should satisfy. We then formally define our metrics and give their
mathematical analysis. We have developed a practical based framework to
maintain the evolving large-scale network and the algorithm to compute our
metrics with preferable time and space consumption, including historical
queries. Finally, we put the result of experiment with a couple of authentic
data.

\smallskip
\noindent \textbf{Keywords:} social network, dynamic graph, centrality,
activity, algorithm.
\end{abstract}
\tableofcontents
\newpage

\section{Introduction}
% TODO
\subsection{Dynamic Metrics}
% TODO
\subsection{Algorithms}
% TODO
\section{Evolving Social Network}
% TODO
\subsection{Definition}
% TODO
\subsection{Properties}
% TODO
\section{Centrality and Activity}
% TODO
\subsection{Desiderata}
% TODO
\subsection{Design and Analysis}
% TODO
\section{Algorithms}
% TODO
\subsection{Maintaining the basic graph structure}
% TODO
\subsection{Historical Queries}
% TODO
\subsection{Computing Metrics}
% TODO
\subsection{Mining Significant Data}
% TODO
\subsection{Optimization and Parallelizability}
% TODO
\section{Experiment}
% TODO
\section{Conclusion and Perspective}
% TODO
\nocite{*}
\bibliography{paper.bib}
\end{document}

% vim: set ts=8 sw=8 noet sts=8 tw=79 indentexpr= fdm=marker foldmarker=<<<,>>>:
