\documentclass[12pt,abstract=true]{scrartcl}

%<<< Preamble
\usepackage{tikz}
\usepackage{array}
\usepackage{pifont}
\usepackage{engord}
\usepackage{listings}
\usepackage{titlesec}
\usepackage{fancyhdr}
\usepackage{fancyvrb}
\usepackage{multicol}
\usepackage{pgfplots}
\usepackage{algorithm}
\usepackage{algpseudocode}
\usepackage[plain]{fancyref}
\usepackage[unicode]{hyperref}
\usepackage{graphicx,tabularx}
\usepackage[shortlabels]{enumitem}
\usepackage[retainorgcmds]{IEEEtrantools}
\usepackage{amsmath,amssymb,amsthm,amscd,amstext}
\usepackage{mathtools}
\usepackage{exscale}
\usepackage{relsize}
\usepackage{mathrsfs}
\usepackage{tocloft}

\usepackage{setspace}
\usepackage{lastpage}
\usepackage{extramarks}
\usepackage{chngpage}

\usepackage{lmodern}
\usepackage{mathpazo}

\pgfplotsset{compat=1.6}
\usetikzlibrary{arrows}
\usetikzlibrary{patterns}
\usetikzlibrary{positioning}
\usetikzlibrary{automata}
\usepackage{tikz-3dplot}

\tikzset{ alert/.style={ very thick, draw=red!80, fill=red!20 } }
\tikzset{ fancy/.style={ very thick, draw=blue!80, fill=blue!20 } }
\tikzset{
  dot/.style={
    very thick,circle,draw=black,fill=black,inner sep=0,minimum size=5pt
  }
}
\tikzset { every state/.style={fancy} }
\tikzset {
  general/.style = {
    shorten >=2pt, node distance=2.5cm, on grid, thick,>=stealth', auto
  }
}
\newcommand{\mathhuge}[1]{\mathlarger{\mathlarger{\mathlarger{#1}}}}
\newcommand{\mathtiny}[1]{\mathsmaller{\mathlarger{\mathlarger{#1}}}}

\renewcommand\cftsecfont{\normalsize}
\renewcommand\cftsecpagefont{\normalsize}
\renewcommand\cftsecafterpnum{\par}
\setlength{\cftbeforesecskip}{0.3em}

\renewcommand\cftsubsecfont{\small}
\renewcommand\cftsubsecpagefont{\small}
\renewcommand\cftsubsecafterpnum{\par}

\newcommand\upper[1]{\textsuperscript#1}
\numberwithin{equation}{section}
\mathtoolsset{centercolon}
\bibliographystyle{plain}

% Theorem environment <<<
\theoremstyle{definition}   \newtheorem{definition}{Definition}[section]
\theoremstyle{plain}        \newtheorem{theorem}{Theorem}[section]
\theoremstyle{plain}        \newtheorem{observation}{Observation}[section]
\theoremstyle{plain}        \newtheorem{fact}{Fact}[section]
\theoremstyle{plain}        \newtheorem{claim}{Claim}[section]
\theoremstyle{plain}        \newtheorem{lemma}[theorem]{Lemma}
\theoremstyle{plain}        \newtheorem{corollary}[theorem]{Corollary}
\theoremstyle{remark}       \newtheorem{example}{Example}[section]
\theoremstyle{remark}       \newtheorem{remark}{Remark}[section]
\newcommand*{\fancyrefdeflabelprefix}{def}
\newcommand*{\fancyrefthmlabelprefix}{thm}
\newcommand*{\fancyreflemlabelprefix}{lem}
\newcommand*{\fancyrefcorlabelprefix}{cor}
\newcommand*{\fancyrefalglabelprefix}{alg}
\newcommand*{\fancyreflnlabelprefix}{ln}
\frefformat{plain}{\fancyreflnlabelprefix}{line #1}
\Frefformat{plain}{\fancyreflnlabelprefix}{Line #1}
\frefformat{plain}{\fancyrefalglabelprefix}{algorithm (#1)}
\Frefformat{plain}{\fancyrefalglabelprefix}{Algorithm (#1)}
\frefformat{plain}{\fancyrefdeflabelprefix}{definition (#1)}
\Frefformat{plain}{\fancyrefdeflabelprefix}{Definition (#1)}
\frefformat{plain}{\fancyrefthmlabelprefix}{theorem (#1)}
\Frefformat{plain}{\fancyrefthmlabelprefix}{Theorem (#1)}
\frefformat{plain}{\fancyreflemlabelprefix}{lemma (#1)}
\Frefformat{plain}{\fancyreflemlabelprefix}{Lemma (#1)}
\frefformat{plain}{\fancyrefcorlabelprefix}{corollary (#1)}
\Frefformat{plain}{\fancyrefcorlabelprefix}{Corollary (#1)}
%>>>
% Alter some LaTeX Float for better treatment of figures: <<<
% See p.105 of "TeX Unbound" for suggested values.
% See pp. 199-200 of Lamport's "LaTeX" book for details.
%   General parameters, for ALL pages:
\renewcommand{\topfraction}{0.9}	% max fraction of floats at top
\renewcommand{\bottomfraction}{0.8}	% max fraction of floats at bottom
\setcounter{topnumber}{2}
\setcounter{bottomnumber}{2}
\setcounter{totalnumber}{4}     % 2 may work better
\setcounter{dbltopnumber}{2}    % for 2-column pages
\renewcommand{\dbltopfraction}{0.9}	% fit big float above 2-col. text
\renewcommand{\textfraction}{0.07}	% allow minimal text w. figs
\renewcommand{\floatpagefraction}{0.7}	% require fuller float pages
\renewcommand{\dblfloatpagefraction}{0.7}	% require fuller float pages

\allowdisplaybreaks[4]

%>>>
%>>>

\begin{document}

\begin{center} %<<< Title
	\Large \textbf{\textsf{
		Metrics and Algorithms for Evolving Social Networks}} \\[1.2em]
	\normalsize 
		Pufan He\upper{1}, Yichao Zhou\upper{2}, Qiwei Feng\upper{3},
		Yu Yan\upper{4}, Qingyang Li\upper{5} and Xin Xing\upper{6}\\[1em]
	\small
	\begin{tabular}{*{3}{>{\centering}p{.3\textwidth}}}
		\upper{1}\small\href{mailto:hpfdf@126.com}{hpfdf@126.com} &%
		\upper{2}\href{mailto:broken.zhou@gmail.com}{broken.zhou@gmail.com} &%
		\upper{3}\href{mailto:gdfqw93@163.com}{gdfqw93@163.com} \tabularnewline
		\upper{4}\href{mailto:whyvine@hotmail.com}{whyvine@hotmail.com} &
		\upper{5}\href{mailto:591527324@qq.com}{591527324@qq.com} &
		\upper{6}\href{mailto:823291634@qq.com}{823291634@qq.com}
	\end{tabular} \\[1em]

	\small Institute for Interdisciplinary Information Sciences,
	Tsinghua University \\[1.5em]
\end{center} \par %>>>

\begin{abstract}
We address the dynamic properties of typical social networks, in which person
is considered as vertex while relation is considered as edge. We suppose the
evolution of a social network overtime only includes bidirectional new
connections between existing or newly created nodes, and we assume the graph
will evolve as a typical social network. We study the activity and centrality
metrics for nodes in such graphs by regulate a series of reasonable properties
the metrics should satisfy. We then formally define our metrics and give their
mathematical analysis. We have developed a practical based framework to
maintain the evolving large-scale network and the algorithm to compute our
metrics with preferable time and space consumption, including historical
queries. Finally, we put the result of experiment with a couple of authentic
data.

\smallskip
\noindent \textbf{Keywords:} social network, dynamic graph, activity,
centrality, algorithm.
\end{abstract}
\tableofcontents
\newpage

\section{Introduction}
Social networks is a surprising interdisciplinary field of social science and
computer science, allowing helpful analysis of large-scaled social behavior
and tendency. In this paper, we will take more information of social networks,
the time dimension, and make the similar study to the social behaviors. We
build metrics for evolving social networks and design the algorithm to
calculate them. We compare how different the outcome by our method would be
against the classical analysis with un-timestamped data mining.

\subsection{Dynamic Metrics}
Networks contain too much information to be easily observed by people without
any abstraction. To study a network, people exploited the basic form of
quantification to concentrate on specific features of the whole graph, or an
individual vertex. Our work is to promote this methodology to the dynamic
graphs. We basically design the new dynamic metrics according to existing
successful static ones, but we will point out a important metric that can be
only observed in time evolving networks, the activity. Intuitively speaking,
a vertex with high activity indicates more frequent recent social actions, and
the vertex is more willing to make new change by the structural sense of graph.
The accuracy of activity will be quite vital to recommendation algorithms of
advertisement, new friends, or search results. Activity can be also used by
other metrics, so that the computation of other metrics will no longer need the
time information directly, but referring to the activity instead.

Centrality, usually defined on vertices, is frequently used to measure the
importance of a specific node in graphs of all kinds. There are many previous
studies about the static graph centralities[?], and there are many useful
network analysis algorithms that used some kind of graph centralities. However,
most social networks change over time, and even more unfortunately, some of the
networks are large-scaled, which could make the analysis become hard and
costly, hence limit the scale of meaningful data mining. Moreover, the time
effect can also become a factor of the centrality, that is, the same graph with
different evolving sequence may have totally different metrics. On an
intuitive level, the node connected with more higher activity nodes should be
more important at that time, even if other node have much more silent
neighbors. If we do not excavate the information on the time dimension, we will
lose tremendous information and also the possibility of accumulative
computation which could be helpful in enhancement the speed of answering
consistent quering.

In Section 3 we will concentrate on centrality and activity in evolving social
network. We will define what centrality or activity is good, and we will design
some instance of those metrics by the standard we set. In Section 4, we will
give the algorithms to compute our designed centrality and activity from the
online data structure which maintains the evolving social network. We will also
show how to quickly locate significant vertices that has high activity or
centrality at a specific time. In Section 5, we will show the metrics make
sense with real world data.

\subsection{Algorithms}
Graph theory is very successful on storing and solving variant problems on
static networks, we have sophisticated methodology on finding the shortest
paths, stratch a minimal spanning tree, finding the maximal $S$-$T$ flow, etc.
And we also have a well-established theoretical system to study problems that
do not seems to have perfect solutions. However, dynamic online queries can be
very hard for general graphs. Best known solutions for many dynamic basic graph
problems still remain unsatisfiable for general demands. So to design metrics
for an general evolving network is a very challenging and open problem. We
concentrate on the popular studied social networks and try to create meaningful
and available approaches of operation on such networks.

In Section 2 we will discuss the known properties of a typical social network,
which could be helpful for our algorithm designs. In Section 4, we will
establish a framework of data structure and algorithms to solve series of
possible queries including computing the metrics about a evolving social
network.

\section{Evolving Social Network}
In some related research on evolving graphs, the time dimension is represented
by constructing a sequence of graphs $EG=\{G_t(V_t,E_t)\}$ on different time
spots $t$. This model is highly expressive and universal in mathematical
statement, thus we adopt in. However, we constrain the relation of
time-adjacent graphs in order to simplify the analysis and remain the highest
possible loyalty to the reality. We build the \textbf{pure growing model},
which care about the new connection forming among vertices, neglecting the
anti-growing operations such as edge deletions. We assume the existence of
every connection will contribute to the metrics, even if the connection is
later erased, so we do not delete edges or nodes in our model, instead we
could add an decay mechanism to reduce the affect of old formed links, which
will be discussed in the next section. Note that in pure growing model, we
will assign every edge a timestamp, namely when the edge is created. And thus
there could be more than one edges between a pair of nodes, with different
timestamp.

\subsection{Definition}
\definition A \textbf{pure growing model} for an evolving graph $EG$ is defined
in pair \begin{equation}
EG=(V,E),
\end{equation}
where $V$ is the set of all nodes in the network, and $E$
is the set of edges, formally an edge $e\in E$ for $e=(u,v,t)$,
where $u,v\in V$ are the two end points of the edge, $u<v$ (to ensure
bidirectional), $t\in \mathbb{N}$. The $t$ component stored the forming time of
the graph. As for weighted evolving graph model, we modifies $e\in E$ for
$e=(u,v,t,w)$, where $w\in\mathbb{R}^+$ stands for the multiplicity of the
edge, which might be useful in some circumstances, and the others stay the same
meaning. In the unweighted case, we can simply set $w=1$ for all edges.

\definition The historical edge set $E_t$
\begin{equation}
E_t=\{e\;|\;e(u,v,t',w)\in E\text{ and }t'\leq t\}.
\end{equation}
An immediate statement one can ensure is
\begin{equation}
\forall t, E_t\subseteq E\text{ and }p\leq q\implies E_p\subseteq E_q.
\end{equation}

Define $G_t$ to be the graph formed by $V$ and $E_t$. So $G_t$ is the snapshot
of the whole network at time $t$.

Now we define the update of an evolving graph in our model. First we assume all
possible active vertices has been already defined, so $V$ does not change.
\definition 
$EG^*=(V,E^*)$ where $E\subseteq E^*$. We assume the edges are added to the
model by correct time order, so
\begin{equation}
\forall e\in E^*\setminus E, \forall e'\in E, e=(u,v,t,w)\text{ and }
e'=(u',v',t',w'), t\geq t'.
\end{equation}

And we define the notation of some frequently used basic functions:

\definition The adjacency matrix.
\begin{equation}
A(G_t)=\begin{pmatrix}a_{ij}\end{pmatrix}_{n\times n},\text{ where }n=|V|,
a_{ij}\in \mathbb{R}^+,
\end{equation}
is the adjacency matrix of the graph at time $t$. More detailed
\begin{equation}
a_{ij}=\begin{cases}
0&(i=j)\\
\sum_{e\in E_t\text{ and }e=(i,j,t,w)}w &(i<j)\\
a_{ji}&(i>j)
\end{cases}
\end{equation}


\begin{definition}
The degree function $k(G_t,v)$.
\begin{equation}
k(G_t,v)=\sum_{\mathclap{e\in E_t, e=(x,y,t,w)}}w(1-\delta(x,v)\delta(y,v)),
\end{equation}
where \begin{equation}
\delta(x,y)=\begin{cases}1&(x\neq y)\\0&(x=y)\end{cases}.
\end{equation}
\end{definition}

\subsection{Properties}
% TODO
\section{Activity and Centrality}
% TODO
\subsection{Desiderata}
% TODO
\subsection{Design and Analysis}
% TODO
\section{Algorithms}
% TODO
\subsection{Maintaining the basic graph structure}
% TODO
\subsection{Historical Queries}
% TODO
\subsection{Computing Metrics}
% TODO
\subsection{Mining Significant Data}
% TODO
\subsection{Optimization and Parallelizability}
% TODO
\section{Experiment}
% TODO
\section{Conclusion and Perspective}
% TODO
\nocite{*}
\bibliography{paper.bib}
\end{document}

% vim: set ts=8 sw=8 noet sts=8 tw=79 indentexpr= fdm=marker foldmarker=<<<,>>>:
